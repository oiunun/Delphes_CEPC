\documentclass[11pt,a4paper]{define/cepcnote}
%\documentclass[coverpage]{define/cepcnote} 
\graphicspath{{figures/}}
\usepackage{define/cepcphysics}
\usepackage{subfigure}
\usepackage{mathrsfs}
\usepackage{authblk}



\title{The CEPC {\tt FSClasser} Package\\ (version 00-00-00, tag dev)}
\author{Gang Li}
\mail  {ligang@mail.ihep.ac.cn}
%\draftversion{1.0}
\cepcnote{CEPC\_TLS\_HIG\_2015\_00X}
\abstracttext{
	This document describes how to use the {\tt FSClasser} package to simultaneously reconstruct 
	and classify multiple final states, as well as how to customize the reconstruction processors within the Marlin framework.  
	Final states can be reconstructed either exclusively or inclusively.  
	The output of the {\tt FSClasser} package are some simple plots and a set of root trees that can then be used as the basis for further analysis.
}


\parindent 0pt
\parskip 10pt

\begin{document}

\tableofcontents
\clearpage

\section{Purpose}

{\tt FSClasser} is a package that can be used to convert CEPC$(lcio)$ data into a format that can be more easily analyzed.  
{\tt FSClasser} ``{\bf Filter}s'' and ``{\bf Classify}s'' information from many ``{\bf F}inal {\bf S}tates'' into root trees.  
Any number of final states ($<300$) can be selected using input parameters in the job options xml-file 
(a file to control programs running in the CEPC {\tt Marlin} framework).  
The final states can be reconstructed either exclusively 
(in which case a kinematic fit to the initial four-momentum is performed by default) 
or inclusively (in which case a range of missing energy/momentum/mass can be specified).  
Only very loose cuts are applied at this stage --- it is assumed that most of the analysis will be done using the output root trees.

\section{Overview}

The {\tt FSClasser} algorithm follows four basic steps:

\begin{enumerate}
\item{Lists of objects are made according to the loose selection criteria described in section~\ref{sec:particles}.  
	   The objects considered are $e^{\pm}$, $\mu^{\pm}$, $\tau^{\pm}$, photon and Jets. 
      Tracks and showers/clusters of objects can appear in multiple lists at this stage.}
\item{The object lists are combined to form a specified final state.  
	   All combinations are considered.  
		Here, particles and jets can only be used once per combination to avoid multi-counting.}
\item{If a final state is being exclusively reconstructed, 
	   a kinematic fit to the initial four-momentum is performed.  
		For inclusive reconstruction, missing energy/momentum/mass squared cuts can be specified.  
		This is described further in section~\ref{sec:event}.}
\item{Finally, the resulting information is written out to a series of root trees, described in section~\ref{sec:tree}.}
\end{enumerate}


In the following documentation, the object reconstruction\&selection criteria~(section~\ref{sec:particles}) 
and event selection criteria (section~\ref{sec:event}) are first discussed.  
Then a number of details are given regarding the final state numbering scheme~(section~\ref{sec:numbering}) 
and the job options files~(sections~\ref{sec:parameters} and~\ref{sec:example}).  
The contents of the output root trees are listed in section~\ref{sec:tree}.  

\section{Particle Selection}
\label{sec:particles}

The {\tt FSClasser} algorithm begins by creating lists of final state objects according to the selection criteria 
discussed in this section.

\subsection[Isolated lepton ($e^{\pm} and \mu^{\pm}$)]{\boldmath Isolated lepton ($e^{\pm} and \mu^{\pm}$)}
\label{sec:tracks}

Leptons are selected based on the $ReconstructedParticles$ collection from either $Arbor$ or $Pandora$. 


The Ecal/Hcal information is used for the paticle identification. 
The requirement on $E_{cone}/E_{lepton}$ is used to insure the lepton candidates are isolated, which can be tuned in the xml config file. 


\subsection[Showers ($\gamma$)]{\boldmath Showers ($\gamma$)}
\label{sec:showers}

Showers are reconstructed using the {\tt RecEmcShower} package.  
In addition to the requirements imposed by the {\tt RecEmcShower} package, 
the following standard selection criteria must be satisfied:

\begin{verbatim}
    (1) 0 < shower time < 14
    (2) |cos(theta)| < 0.80 (barrel photons) 
           or 0.86 < |cos(theta)| < 0.92 (endcap photons)
    (3) for barrel photons:  E > 25 MeV
    (4) for endcap photons:  E > 50 MeV
\end{verbatim}

Note that these additional requirements can be turned off using the {\tt neutralStudies} parameter (see section~\ref{sec:neutralparameter}).

\subsection[$\pi^0\to\gamma\gamma$ and $\eta\to\gamma\gamma$]{\boldmath $\pi^0\to\gamma\gamma$ and $\eta\to\gamma\gamma$}
\label{sec:gg}

Lists of $\pi^0\to\gamma\gamma$ and $\eta\to\gamma\gamma$ decays are made using the {\tt EvtRecPi0} and  {\tt EvtRecEtaToGG} packages, respectively.  The following additional cuts are imposed:

\begin{verbatim}
    (1) chi2 < 2500 (from the fit to the eta or pi0 mass)
    (2) for pi0:  0.107 < mass(gamma gamma) < 0.163 GeV/c2
        for eta:  0.400 < mass(gamma gamma) < 0.700 GeV/c2
\end{verbatim}

These default cuts can be removed using the {\tt pi0Studies} or {\tt etaStudies} job options parameters~(sections~\ref{sec:pi0parameter} and~\ref{sec:etaparameter}).  Both photons from the $\pi^0$ and $\eta$ are also required to pass the shower cuts listed in section~\ref{sec:showers}.


\subsection[$K^0_S \to \pi^+\pi^-$, $\Lambda \to p\pi^-$ and $\overline{\Lambda}\to\overline{p}\pi^+$]{\boldmath $K^0_S \to \pi^+\pi^-$, $\Lambda \to p\pi^-$ and $\overline{\Lambda}\to\overline{p}\pi^+$ }
\label{sec:vee}

$K^0_S \to \pi^+\pi^-$, $\Lambda \to p\pi^-$ and $\overline{\Lambda}\to\overline{p}\pi^+$ reconstruction is done with the {\tt EvtRecVeeVertex} package with the following additional cuts:

\begin{verbatim}
    (1) chi2 < 100 (from the fit to the Ks or Lambda vertex)
    (2) for Ks:      0.471 < mass(pi+ pi-) < 0.524 GeV/c2
        for Lambda:  1.100 < mass(p pi) < 1.300 GeV/c2
\end{verbatim}

These default cuts can be removed with the {\tt veeStudies} job options parameter~(section~\ref{sec:veeparameter}).

\section{Event Selection}
\label{sec:event}

Once the lists of particles are made, they are combined to form different final states.  All combinations are considered.  To avoid double-counting, tracks and showers (including tracks and showers from $\pi^0$, $\eta$, $K^0_S$, and $\Lambda$ decays) are used only once per combination.  Parameters can be specified in the job options file (see section~\ref{sec:parameters}) to reduce the combinatorics.

For both inclusive and exclusive final states, a primary vertex fit (using the {\tt VertexFit} package) is performed using tracks originating from the primary vertex.  The primary vertex and the resulting track parameters are saved.  If the final state contains no tracks from the primary vertex, the beam spot is used as the primary vertex and no fit is performed.  Secondary vertices due to $K^0_S$ and $\Lambda$ decays are then created with the {\tt SecondVertexFit} package.

For inclusive final states, bounds on the missing mass or a kinematic fit to the missing mass (using the {\tt KalmanKinematicFit} package) can be specified in the job options file (see section~\ref{sec:parameters}).  For exclusive final states, the {\tt KalmanKinematicFit} package is used to fit the four-momenta of the final state particles to the initial four-momentum of the $e^+e^-$ interaction.  In either case, additional constraints are placed on the $\pi^0$, $\eta$, $K^0_S$, and $\Lambda$ masses.  A loose cut (that can be tuned in the job options file) is placed on the resulting $\chi^2$.  Multiple combinations can be recorded per event.


\section{Final State Numbering}
\label{sec:numbering}

{\tt FSClasser} can reconstruct any final state composed of a combination of $\Lambda$ (called {\tt Lambda}, decaying to $p\pi^-$), $\overline\Lambda$ (called {\tt ALambda}, decaying to $\overline{p}\pi^+$), $e^+$, $e^-$, $\mu^+$, $\mu^-$, $p^+$~(proton), $p^-$~(anti-proton), $\eta(\to\gamma\gamma)$, $\gamma$, $K^+$, $K^-$, $K^0_S(\to\pi^+\pi^-)$, $\pi^+$, $\pi^-$, and $\pi^0(\to\gamma\gamma)$.  Final states are designated using two integers, ``code1'' and ``code2''.  The digits of each integer are used to specify the number of different particle types in the final state:
\begin{verbatim}
    code1 = abcdefg
            a = number of gamma
            b = number of tau+
            c = number of tau-
            d = number of mu+
            e = number of mu- 
            f = number of e+  
            g = number of e- 

    code2 = h
            h = number of jets                
\end{verbatim}

These integers are sometimes combined into a single string of the form:

\begin{verbatim}
    "code2_code1"
\end{verbatim}

Here are a few examples:

\begin{verbatim}
    "0_11":       e+ e-         
    "0_1000002":  gamma e+ e-   
    "2_001100":   jet jet mu+ mu-
    "0_0111100":  tau+ tau- mu+ mu-
\end{verbatim}

\section{Job Options Parameters}
\label{sec:parameters}

\subsection{\tt FSClasser.FS(N) = "(EXC/INC)(tag)(code2)\_(code1)";}
\label{sec:fsn}

Use a list of these commands to specify which final states to reconstruct.
\begin{verbatim}
    (N): a unique integer between 0 and 9999 for each 
         reconstructed final state (this is only used to 
         differentiate FS(N) commands, and isn't used
         elsewhere)

    (EXC/INC): use EXC to reconstruct the final state 
               exclusively (using a kinematic fit);
               use INC for inclusive reconstruction

    (tag):  an optional string to help further distinguish
            final states (for example, use this if you want
            to apply distinct sets of cuts to the same
            final state)

    (code2) and (code1):  the final state numbering integers 
                          described in the "final state 
                          numbering" section
\end{verbatim}
Examples:
\begin{verbatim}
    Reconstruct eta K+ K- exclusively:

        FSClasser.FS100 = "EXC1_110000";

    Reconstruct pi+ pi- inclusively:

        FSClasser.FS200 = "INC0_110";
\end{verbatim}


\subsection{\tt FSClasser.FSCut(N) = "(FS) (submode) (type) (x1) (x2)";}

The {\tt FSCut} parameters can be used to place cuts on intermediate particle masses or recoil masses.  
This can dramatically decrease file sizes when studying final states with intermediate resonances.

\begin{verbatim}
    (N): a unique integer between 0 and 9999 for each 
         cut (this is only used to differentiate cuts, 
         and isn't used elsewhere)

    (FS): the name of the final state in which this cut
          should be applied (this corresponds to the
          "(EXC/INC)(tag)(code2)_(code1)" name of the
          previous section) 

    (submode): the particle subsystem that will be used
               in the cut, with the format "(code2)_(code1)"
               where the codes are described in the
               "final state numbering" section

    (type): the type of cut -- options are:
            "RawMass", "IntMass", "FitMass",
               (for mass selections using raw, intermediate
                 or fitted four-vectors) 
            "RawRecoil", "IntRecoil", "FitRecoil",
               (for recoil mass selections)
            or any of the above with "Squared" appended
               (for mass squared or recoil mass squared)

    (x1):  the lower limit of the cut in GeV or GeV^2

    (x2):  the upper limit of the cut in GeV or GeV^2
\end{verbatim}

Examples:

\begin{verbatim}
    To select phi eta from the K+ K- pi+ pi- pi0 final state:

        FSClasser.FS1 = "EXC0_110111";
        FSClasser.FSCut11 = "EXC0_110111 0_110000 FitMass 0.9 1.1";
        FSClasser.FSCut21 = "EXC0_110111 0_111    FitMass 0.4 0.7";

    To look for psi(2S) --> pi+ pi- J/psi inclusively:

        FSClasser.FS2 = "INC0_110";
        FSClasser.FSCut12 = "INC0_110 0_110 RawRecoil 3.0 3.2";
\end{verbatim}
In cases where multiple intermediate particle combinations can be formed in a given final state, all combinations are checked, and the event passes if any of the combinations pass the cut.


\subsection{\tt FSClasser.FSMMFit(N) = "(FS) (mass)";}

(INCLUSIVE RECONSTRUCTION ONLY) For inclusive final states, perform a kinematic fit to the missing mass~(MM) using the {\tt FSMMFit} parameters.
\begin{verbatim}
    (N): a unique integer between 0 and 9999 for each 
         cut (this is only used to differentiate parameters, 
         and isn't used elsewhere)

    (FS): the name of the final state in which this cut
          should be applied (this corresponds to the
          "(EXC/INC)(tag)(code2)_(code1)" name of the
          FS(N) parameter) 

    (mass): the mass to which the missing mass will be
            constrained (for example, the neutron mass)
\end{verbatim}
Example:
\begin{verbatim}
    To reconstruct J/psi --> p+ anti-n pi- with a missing 
    anti-neutron use:

        FSClasser.FS1 = "INC100_10";
        FSClasser.FSMMFit1 = "INC100_10 0.9396";
\end{verbatim}


\subsection{\tt FSClasser.maxShowers = (n);}

Only consider events with less than or equal to {\tt (n)} showers.  The default is 50.  The purpose of this requirement is just to speed up reconstruction.  When there are more than 50~showers in an event, the possible combinations of showers used to form photons or $\pi^0$ or $\eta$ can become very large.


\subsection{\tt FSClasser.maxExtraTracks = (n);}

(EXCLUSIVE RECONSTRUCTION ONLY) For exclusive reconstruction, only allow {\tt (n)} extra tracks (tracks beyond those required in the final state) in the event.  The default is 2.  This requirement is ignored for inclusively reconstructed final states.


\subsection{\tt FSClasser.cmEnergy = (E);}

Specify the center of mass energy~{\tt (E)} in units of GeV.  The default is the $\psi(2S)$ mass, $3.686093~\gev$.  (Eventually the default will come from an external {\tt BeamEnergy} package.)  If this is set to -1, the run number is used to set the center of mass energy, but this currently only works for a limited number of run ranges.

\subsection{\tt FSClasser.crossingAngle = (theta);}

Set the crossing angle in radians.  The default is $0.011$~radians.

\subsection{\tt FSClasser.energyTolerance = (deltaE);}

(EXCLUSIVE RECONSTRUCTION ONLY) For exclusive reconstruction, only combinations of final state particles that have a total energy within {\tt (deltaE)} of the initial total energy will be processed.  {\tt (deltaE)} is given in~GeV.  This reduces the number of kinematic fits that are performed and helps speed processing.  The default is $0.250~\gev$.

\subsection{\tt FSClasser.momentumTolerance = (deltaP);}

(EXCLUSIVE RECONSTRUCTION ONLY) Like {\tt energyTolerance}, this command requires the total momentum of final state particles be within {\tt (deltaP)} (in~$\gevc$) of the initial total momentum.  The default is $0.250~\gevc$.

\subsection{\tt FSClasser.maxKinematicFitChi2DOF = (chi2DOF);}

For exclusive reconstruction, or inclusive reconstruction that includes a kinematic fit, only record events that have a kinematic fit $\chi^2/dof$ less than {\tt (chi2DOF)}.  Using this can reduce file sizes.  The default is $100$.


\subsection{\tt FSClasser.trackStudies = (true/false);}
\label{sec:trackparameter}

If {\tt true}, remove all of the default track selection criteria listed in section~\ref{sec:tracks}.  The default is {\tt false}.


\subsection{\tt FSClasser.pidStudies = (true/false);}
\label{sec:pidparameter}

If {\tt true}, include additional PID information in the root tree.  The default is {\tt false}.


\subsection{\tt FSClasser.neutralStudies = (true/false);}
\label{sec:neutralparameter}

If {\tt true}, remove all of the default photon selection criteria listed in section~\ref{sec:showers}.  The default is {\tt false}.

\subsection{\tt FSClasser.pi0Studies = (true/false);}
\label{sec:pi0parameter}

If {\tt true}, remove all of the default $\pi^0$ selection criteria listed in section~\ref{sec:gg}.  The default is {\tt false}.

\subsection{\tt FSClasser.etaStudies = (true/false);}
\label{sec:etaparameter}

If {\tt true}, remove all of the default $\eta$ selection criteria listed in section~\ref{sec:gg}.  The default is {\tt false}.

\subsection{\tt FSClasser.veeStudies = (true/false);}
\label{sec:veeparameter}

If {\tt true}, remove all of the default $K_S$ and $\Lambda$ selection criteria listed in section~\ref{sec:vee}.  The default is {\tt false}.

\subsection{\tt FSClasser.bypassVertexDB = (true/false);}

If {\tt true}, zero the beam vertex information.  This is useful for testing data before vertex information is available.  The default is {\tt false}.


\subsection{\tt FSClasser.YUPING = (true/false);}

If {\tt true}, use the kaon and pion tracking corrections developed by Yuping Guo.  The default is {\tt false}.

\subsection{\tt FSClasser.writeNTGen = (true/false);}

If {\tt true}, write the global truth information tree ({\tt ntGEN}, described in section~\ref{sec:tree}).  The default is {\tt true}.  If you have already run over MC, setting this to {\tt false} can reduce file sizes.

\subsection{\tt FSClasser.writeNTGenCode = (true/false);}

If {\tt true}, write the truth information associated with every mode~(the tree named {\tt ntGEN(tag)(code2)\_(code1)}, described in section~\ref{sec:tree}).  The default is {\tt true}.  

\subsection{Parameters for Debugging}

\subsubsection{\tt FSClasser.printTruthInformation = (true/false);}

If {\tt true}, print MC truth information to the screen.  This can be useful for debugging.  The default is {\tt false}.

\subsubsection{\tt FSClasser.isolateRunLow = (run);}

Only process runs with run number greater than or equal to {\tt run}.  This also turns on some debugging output.  The default is $-1$, which means it is not used.

\subsubsection{\tt FSClasser.isolateRunHigh = (run);}

Only process runs with run number less than or equal to {\tt run}.

\subsubsection{\tt FSClasser.isolateEventLow = (event);}

Only process events with event number greater than or equal to {\tt event}.

\subsubsection{\tt FSClasser.isolateEventHigh = (event);}

Only process events with event number less than or equal to {\tt event}.


\section{Example Job Options File}
\label{sec:example}

Here is a job options file to exclusively reconstruct $\psi(2S)\to\gamma X_i$, where $X_i$ is:
\begin{enumerate}
\item{$\pi^+\pi^+\pi^-\pi^-$}
\item{$\pi^+\pi^+\pi^-\pi^-\pi^0\pi^0$}
\item{$\pi^+\pi^+\pi^+\pi^-\pi^-\pi^-$}
\item{$K^-K^0_S\pi^+$}
\item{$K^+K^0_S\pi^-$}
\item{$K^+K^-\pi^0$}
\item{$K^+K^-\pi^+\pi^-$}
\item{$K^-K^0_S\pi^+\pi^+\pi^-$}
\item{$K^+K^0_S\pi^+\pi^-\pi^-$}
\item{$K^+K^-\pi^+\pi^-\pi^0$}
\item{$K^+K^+\pi^+\pi^+\pi^-\pi^-$}
\item{$K^+K^+K^-K^-$}
\item{$\eta\pi^+\pi^-$}
\item{$\eta\pi^+\pi^+\pi^-\pi^-$}
\end{enumerate}
and to search for $\psi(2S)\to\gamma\eta_c(1S)$ using inclusive photons:

\begin{verbatim}

  // ************************************************
  //      EXAMPLE FSFILTER JOB OPTIONS FILE
  // ************************************************

  #include "$ROOTIOROOT/share/jobOptions_ReadRec.txt"
  #include "$VERTEXFITROOT/share/jobOptions_VertexDbSvc.txt"
  #include "$MAGNETICFIELDROOT/share/MagneticField.txt"
  #include "$ABSCORROOT/share/jobOptions_AbsCor.txt"
  #include "$PI0ETATOGGRECALGROOT/share/jobOptions_Pi0EtaToGGRec.txt"
  #include "$FSFILTERROOT/share/jobOptions_FSClasser.txt"

  FSClasser.cmEnergy = 3.686093;

  FSClasser.FS1  = "EXC0_1000220";
  FSClasser.FS2  = "EXC0_1000222";
  FSClasser.FS3  = "EXC0_1000330";
  FSClasser.FS4  = "EXC0_1011100";
  FSClasser.FS5  = "EXC0_1101010";
  FSClasser.FS6  = "EXC0_1110001";
  FSClasser.FS7  = "EXC0_1110110";
  FSClasser.FS8  = "EXC0_1011210";
  FSClasser.FS9  = "EXC0_1101120";
  FSClasser.FS10 = "EXC0_1110111";
  FSClasser.FS11 = "EXC0_1110220";
  FSClasser.FS12 = "EXC0_1220000";
  FSClasser.FS13 = "EXC1_1000110";
  FSClasser.FS14 = "EXC1_1000220";

  FSClasser.FS100 =    "INC0_1000000";
  FSClasser.FSCut101 = "INC0_1000000 0_1000000 RawRecoil 2.6 3.2";

  // Input REC or DST file name 
  EventCnvSvc.digiRootInputFile = {"/bes3fs/offline/...."};

  // Set output level threshold 
  // (2=DEBUG, 3=INFO, 4=WARNING, 5=ERROR, 6=FATAL )
  MessageSvc.OutputLevel = 5;

  // Number of events to be processed (default is 10)
  ApplicationMgr.EvtMax = 500;

  ApplicationMgr.HistogramPersistency = "ROOT";
  NTupleSvc.Output = { "FILE1 DATAFILE='PsiPrimeGammaX.root' 
                       OPT='NEW' TYP='ROOT'"};
\end{verbatim}


\section{Output Root Trees}
\label{sec:tree}

{\tt FSClasser} generates three types of root trees:
\begin{enumerate}
\item{Reconstructed information is stored in trees named using the convention described in section~\ref{sec:fsn} (with an {\tt nt} prepended): {\tt nt(EXC/INC)(tag)(code2)\_(code1)}, where {\tt (EXC/INC)} is either {\tt EXC} or {\tt INC} depending on whether the reconstruction was done exclusively or inclusively, and {\tt (code2)} and {\tt (code1)} are the numbering codes described in section~\ref{sec:numbering}, etc. There is one entry for each combination of particles that passes the loose cuts described in sections~\ref{sec:particles} and~\ref{sec:event}. Thus there may be more than one entry per event.  This is the main tree for physics analysis.}
\item{MC generated information for each exclusive final state is stored in trees named {\tt ntGEN(tag)(code2)\_(code1)}.  This type of tree contains exactly one entry per each event generated with this final state, regardless of whether or not the event was reconstructed.  It is only used for exclusive final states.  This tree can be used to look at generator-level distributions.}
\item{MC generated information for all final states is stored in one tree named {\tt ntGEN}.  This tree contains exactly one entry per event, independent of final state, and independent of reconstruction.  Use this tree to count the number of events generated for different processes.}
\end{enumerate}


Each of these trees contains some subset of the blocks of information described in the following subsections.  Only the reconstructed tree, {\tt nt(EXC/INC)(tag)(code2)\_(code1)}, for example, contains shower or track information.

By convention particles are listed in trees in the following order:  
\begin{verbatim}
    Lambda ALambda e+ e- mu+ mu- p+ p- eta gamma K+ K- Ks pi+ pi- pi0
\end{verbatim}
For example, in the process $\psi(2S)\to\pi^+\pi^-J/\psi; J/\psi\to\mu^+\mu^-$, the $\mu^+$ is particle~1, the $\mu^-$ is particle~2, the $\pi^+$ is particle~3, and the $\pi^-$ is particle~4. Or as another example, in the process $\psi(2S)\to\gamma\chi_{c1}; \chi_{c1}\to\eta\pi^0\pi^0$, the $\eta$ is particle~1, the $\gamma$ is particle~2, one $\pi^0$ is particle~3, and the other $\pi^0$ is particle~4.  In cases like this where there are identical particles, no ordering is assumed.



\subsection{Event Information}

The ``Event Information'' block contains information about the event as a whole.

The following information is contained in all three tree types:
\begin{verbatim}
  Run:          run number
  Event:        event number
  BeamEnergy:   beam energy
  NTracks:      total number of tracks in the event
  NShowers:     total number of showers in the event
\end{verbatim}
The following information is only contained in {\tt nt(EXC/INC)(tag)(code2)\_(code1)}, where the total energies and momenta are calculated using the raw particle four-momenta (i.e., before any vertex or kinematic fitting):
\begin{verbatim}
  TotalEnergy:    total energy of all final state particles
  TotalPx:        total px of all final state particles
  TotalPy:        total py of all final state particles
  TotalPz:        total pz of all final state particles
  TotalP:         total momentum of all final state particles
  MissingMass2:   missing mass squared of the event
  VChi2:          chi2 of the vertex fit if there was a
                    vertex fit (-1 otherwise)
\end{verbatim}
Finally, this information is only contained in {\tt nt(EXC/INC)(tag)(code2)\_(code1)} and only when a kinematic fit is performed:
\begin{verbatim}
  Chi2:         chi2 of the kinematic fit
  Chi2DOF:      chi2/dof of the kinematic fit
\end{verbatim}


\subsection{Vertex Information}

The ``Vertex Information'' block contains the beam spot and the primary vertex.  It is only written out for the reconstructed tree, {\tt nt(EXC/INC)(tag)(code2)\_(code1)}.

\begin{verbatim}
  BeamVx:      x-position of the beam spot
  BeamVy:      y-position of the beam spot
  BeamVz:      z-position of the beam spot
  PrimaryVx:   x-position of the primary vertex
  PrimaryVy:   y-position of the primary vertex
  PrimaryVz:   z-position of the primary vertex
\end{verbatim}

\subsection{Particle Four-Momenta}
\label{sec:nt:momentum}

The ``Particle Four-Momenta'' block contains the four-momentum for each particle in the final state:

\begin{verbatim}
  (prefix)PxP(n):  x momentum of particle (n)
  (prefix)PyP(n):  y momentum of particle (n)
  (prefix)PzP(n):  z momentum of particle (n)
  (prefix)EnP(n):  energy of particle (n)
\end{verbatim}

Different types of four-momenta are distinguished using prefixes.  MC generated four-momenta have a prefix~{\tt MC}; raw four-momenta have a prefix~{\tt R}; vertex-constrained four-momenta or four-momenta that include 1C mass fits (e.g. constraining the $\pi^0$ or $\eta$ masses) have a prefix {\tt I} (for ``Intermediate''); and the final four-momenta (the fully-constrained four-momenta resulting from the kinematic fit in the exclusive case, or the vertex-constrained and 1C mass-constrained four-momenta in the inclusive case) have no prefix.

Different particles are differentiated using the postfix {\tt P(n)}, where {\tt (n)} is the number of the particle in the ordered list.  Four-momenta for secondaries originating from particle~{\tt (n)}, such as the two $\gamma$'s from a $\pi^0$, are recorded using {\tt P(n)a} and {\tt P(n)b}, where the ordering follows the same conventions as above, or, in the case of identical daughter particles, ordering is from low to high energy.  As two examples: in the process $\psi(2S)\to\pi^+\pi^-J/\psi; J/\psi\to\mu^+\mu^-$, the raw energy of the $\pi^+$ is given by {\tt REnP3}; and in the process $\psi(2S)\to\pi^+\pi^-J/\psi; J/\psi\to\pi^+\pi^-\pi^0$, the y-momentum of the high-energy photon from the $\pi^0$, after a 1C fit to the $\pi^0$ mass, is given by {\tt IPyP5b}.


Particle four-momenta are included in both the {\tt nt(EXC/INC)(tag)(code2)\_(code1)} and the {\tt ntGEN(tag)(code2)\_(code1)} trees.  In the {\tt ntGEN(tag)(code2)\_(code1)} tree, the generated four-momenta are actually written out twice, once with the {\tt MC} prefix and once with no prefix.  This makes it easier to treat the generated MC more like data in some applications.

In the reconstructed tree, {\tt nt(EXC/INC)(tag)(code2)\_(code1)}, MC generated information is also included for events in which the generated and reconstructed final states are the same.  When there are identical particles, however, no attempt is made to ensure that the particles are ordered consistently, i.e., no attempt is made to match reconstructed particles with generated particles.


\subsection{Shower Information}
\label{sec:nt:shower}

Shower information is written out for every reconstructed photon that is part of a final state.  As in the four-momenta case (section~\ref{sec:nt:momentum}), showers from different particles are differentiated using the postfix {\tt P(n)}, where {\tt (n)} is the number of the particle in the ordered list.  As examples: the energy of the low energy photon from the $\pi^0$ decay in $\psi(2S)\to\pi^+\pi^-\pi^0$ is called {\tt ShEnergyP3a}; and the timing of the radiated photon in $\psi(2S)\to\gamma\eta$ is called {\tt ShTimeP2}.
\begin{verbatim}
  ShTimeP(n):     timing information
  ShEnergyP(n):   shower energy
  ShCosThetaP(n): cos(theta) in the lab
  ShE925P(n):     E9/E25 (the energy in a 3x3 array
                    over the energy in a 5x5 array)
  ShPi0PullP(n):  the pull of the best pi0 formed
                    with this shower
  ShDangP(n):     smallest angle between this shower
                    and tracks projected to the EMC
  ShMatchP(n):    1.0 if this shower has an associated
                    charged track; -1.0 otherwise
\end{verbatim}

\subsection{Track Information}
\label{sec:nt:track}

Track information is written out for every reconstructed track that is part of a final state.  The postfix {\tt P(n)} follows the same convention as for the four-momenta (section~\ref{sec:nt:momentum}).
\begin{verbatim}
  TkProbPiP(n):   probability track is a pion
  TkProbKP(n):    probability track is a kaon
  TkProbPP(n):    probability track is a proton
  TkProbMuP(n):   probability track is a muon
  TkProbEP(n):    probability track is an electron
  TkRVtxP(n):     radial distance of closest approach
                    to the primary vertex (in cm)
  TkZVtxP(n):     longitudinal distance
                    to the primary vertex (in cm)
  TkCosThetaP(n): cos(theta) in the lab
  TkEPP(n):       E/p (shower energy over
                    track momentum)
  TkMucDepthP(n): penetration depth in muon chamber
                    (in cm)
\end{verbatim}
If the {\tt pidStudies} flag is set to true, also include these variables:
\begin{verbatim}
  TkProbPHP(n):   a dE/dx variable in arbitrary units
                    (550 for Bhabha electrons)
  TkNormPHP(n):   (not sure)
  TkErrorPHP(n):  (not sure)
  TkIndexP(n):    track index in the track collection
\end{verbatim}

\subsection[$\pi^0\to\gamma\gamma$ Information]{\boldmath $\pi^0\to\gamma\gamma$ Information}
\label{sec:nt:pi0}

The following information for each $\pi^0\to\gamma\gamma$ decay is recorded:
\begin{verbatim}
  Pi0MassP(n):    the unconstrained gamma gamma mass
  Pi0Chi2P(n):    the chi2 of the 1C fit to the pi0 mass
\end{verbatim}
Note that the shower information for each $\gamma$ from the $\pi^0$ is recorded along with the other showers (section~\ref{sec:nt:shower}) and the four-momentum for each photon is recorded along with the other four-momenta (section~\ref{sec:nt:momentum}).

\subsection[$\eta\to\gamma\gamma$ Information]{\boldmath $\eta\to\gamma\gamma$ Information}
\label{sec:nt:eta}

The decay $\eta\to\gamma\gamma$ is treated in the same way as $\pi^0\to\gamma\gamma$ (section~\ref{sec:nt:pi0}):
\begin{verbatim}
  EtaMassP(n):    the unconstrained gamma gamma mass
  EtaChi2P(n):    the chi2 of the 1C fit to the eta mass
\end{verbatim}

\subsection[$K^0_S\to\pi^+\pi^-$, $\Lambda \to p\pi^-$ and $\overline{\Lambda}\to\overline{p}\pi^+$ Information]{\boldmath $K^0_S\to\pi^+\pi^-$, $\Lambda \to p\pi^-$ and $\overline{\Lambda}\to\overline{p}\pi^+$ Information}

This information is recorded for each $K^0_S\to\pi^+\pi^-$, $\Lambda \to p\pi^-$ and $\overline{\Lambda}\to\overline{p}\pi^+$ decay:
\begin{verbatim}
  VeeMassP(n):     the unconstrained mass of the daughters
  VeeChi2P(n):     the chi2 of the initial Ks vertex fit
  Vee2ndChi2P(n):  the chi2 of the secondary vertex fit
  VeeLSigmaP(n):   the separation between the primary and 
                    secondary vertex (L) over its error (sigma)
  VeeSigmaP(n):    the error of the decay length (L)
  VeeVxP(n):       the x-position of the secondary vertex
  VeeVyP(n):       the y-position of the secondary vertex
  VeeVzP(n):       the z-position of the secondary vertex
\end{verbatim}
Note that track information and four-momenta for the daughter tracks are also recorded as in sections~\ref{sec:nt:track} and~\ref{sec:nt:momentum}, respectively.

\subsection[Information to Tag $\psi(2S)\to XJ/\psi$]{\boldmath Information to Tag $\psi(2S)\to XJ/\psi$}

A few variables are included to try to identify $\psi(2S)$ transitions to $J/\psi$, 
sometimes useful for identifying $J/\psi$ backgrounds.  
This information is currently included when running over all data sets, but it is only meaningful for $\psi(2S)$ data.

\begin{verbatim}
  JPsiPiPiRecoil:  the smallest difference between the J/psi
                     mass and any combination of pi+ pi-
                     recoil mass (using all positive and
                     negative tracks with assumed pion 
                     masses)
  JPsiGGRecoil:    the smallest difference between the J/psi
                     mass and any combination of gamma gamma
                     recoil mass
\end{verbatim}

\subsection{Truth Information}

When running over Monte Carlo, the truth information about a given event is included in all three tree types.  
These variables should allow one to separate and count signal and background events.  
The numbers in this block of information always refer to generator level information.  
For example, the ``number of $K^0_L$'' refers to the true number of generated $K^0_L$.

These variables label the generated final state:
\begin{verbatim}
  MCDecayCode1:      the true code1, described in the
                       "final state numbering" section
  MCDecayCode2:      the true code2, described in the
                       "final state numbering" section
  MCExtras:          1000 * number of neutrinos +
                      100 * number of K_L +
                       10 * number of neutrons +
                        1 * number of antineutrons
  MCTotalEnergy:     the total generated energy calculated
                       using particles in MCDecayCode(1,2)
                       and MCExtras:  use this to double-
                       check that no particles are missing
  MCSignal:          1 if the reconstructed final state
                       matches the generated final state;
                     0 otherwise
\end{verbatim}

These variables count open charm particles:

\begin{verbatim}
  MCOpenCharm1:      1000 * number of D+ +
                      100 * number of D- +
                       10 * number of D0 +
                        1 * number of D0bar
  MCOpenCharm2:      1000 * number of (D*+ to pi+ D0) +
                      100 * number of (D*- to pim D0bar) +
                       10 * number of (D*0 to gamma D0) +
                        1 * number of (D*0bar to gamma D0bar)
  MCOpenCharm3:      1000 * number of (D*+ to pi0 D+) +
                      100 * number of (D*- to pi0 D-) +
                       10 * number of (D*0 to pi0 D0) +
                        1 * number of (D*0bar to pi0 D0bar)
\end{verbatim}

These variables keep track of how various states were produced, where {\tt X} stands for any particle:

\begin{verbatim}
  MCChicProduction:  10 for X to gamma chi_c0
                     11 for X to gamma chi_c1
                     12 for X to gamma chi_c2
                      1 for any other production of chi_cJ
                      0 for no produced chi_cJ
  MCJPsiProduction:  1 for X to pi+ pi- J/psi
                     2 for X to pi0 pi0 J/psi
                     3 for X to eta J/psi
                     4 for X to gamma J/psi
                     5 for X to pi0 J/psi
                     6 for any other production of J/psi
                     0 for no produced J/psi
  MCHcProduction:    1 for X to pi+ pi- h_c
                     2 for X to pi0 pi0 h_c
                     3 for X to eta h_c
                     4 for X to gamma h_c
                     5 for X to pi0 h_c
                     6 for any other production of h_c
                     0 for no produced h_c
  MCEtacProduction:  1 for X to gamma eta_c
                     2 for any other production of eta_c
                     0 for no produced eta_c
\end{verbatim}

These variables count the number of ways in which various particles decayed:

\begin{verbatim}
  MCPi0Decay:         10 * number of gamma e+ e- +
                       1 * number of gamma gamma
  MCEtaDecay:       10000 * number of gamma e+ e- +
                     1000 * number of gamma pi+ pi- +
                      100 * number of pi+ pi- pi0 +
                       10 * number of pi0 pi0 pi0 +
                        1 * number of gamma gamma
  MCEtaprimeDecay:  1000000 * number of pi0 pi0 pi0 +
                     100000 * number of gamma gamma +
                      10000 * number of gamma omega +
                       1000 * number of gamma pi+ pi- +
                        100 * number of gamma rho +
                         10 * number of eta pi0 pi0 +
                          1 * number of eta pi+ pi-
  MCPhiDecay:     10000 * number of gamma eta +
                   1000 * number of rho pi +
                    100 * number of pi+ pi- pi0 +
                     10 * number of Ks Kl +
                      1 * number of K+ K-
  MCOmegaDecay:   100 * number of pi+ pi- +
                   10 * number of gamma pi0 +
                    1 * number of pi+ pi- pi0
  MCKsDecay:      10 * number of pi0 pi0 +
                   1 * number of pi+ pi-
  MCFSRGamma:   number of FSR gammas in this event
                 (FSR gammas are ignored elsewhere)
\end{verbatim}

The {\tt MCDecayParticle} variables are an ordered list of the PDG ID numbers of the particles coming from the initial decay.  For example, for $\psi(2S)\to\rho^+\pi^-$, {\tt MCDecayParticle1} is -211 (for the $\pi^-$); {\tt MCDecayParticle2} is 213 (for the $\rho^+$); and all the others are zero.

\begin{verbatim}
  MCDecayParticle1:  the PDG ID of the 1st particle
  MCDecayParticle2:  the PDG ID of the 2nd particle
  MCDecayParticle3:  the PDG ID of the 3rd particle
  MCDecayParticle4:  the PDG ID of the 4th particle
  MCDecayParticle5:  the PDG ID of the 5th particle
  MCDecayParticle6:  the PDG ID of the 6th particle
\end{verbatim}

\end{document}
